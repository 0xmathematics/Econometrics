\documentclass[]{article}
\usepackage{lmodern}
\usepackage{amssymb,amsmath}
\usepackage{ifxetex,ifluatex}
\usepackage{fixltx2e} % provides \textsubscript
\ifnum 0\ifxetex 1\fi\ifluatex 1\fi=0 % if pdftex
  \usepackage[T1]{fontenc}
  \usepackage[utf8]{inputenc}
\else % if luatex or xelatex
  \ifxetex
    \usepackage{mathspec}
  \else
    \usepackage{fontspec}
  \fi
  \defaultfontfeatures{Ligatures=TeX,Scale=MatchLowercase}
\fi
% use upquote if available, for straight quotes in verbatim environments
\IfFileExists{upquote.sty}{\usepackage{upquote}}{}
% use microtype if available
\IfFileExists{microtype.sty}{%
\usepackage{microtype}
\UseMicrotypeSet[protrusion]{basicmath} % disable protrusion for tt fonts
}{}
\usepackage[margin=1in]{geometry}
\usepackage{hyperref}
\hypersetup{unicode=true,
            pdftitle={R tutorial},
            pdfauthor={Hang Miao},
            pdfborder={0 0 0},
            breaklinks=true}
\urlstyle{same}  % don't use monospace font for urls
\usepackage{color}
\usepackage{fancyvrb}
\newcommand{\VerbBar}{|}
\newcommand{\VERB}{\Verb[commandchars=\\\{\}]}
\DefineVerbatimEnvironment{Highlighting}{Verbatim}{commandchars=\\\{\}}
% Add ',fontsize=\small' for more characters per line
\usepackage{framed}
\definecolor{shadecolor}{RGB}{248,248,248}
\newenvironment{Shaded}{\begin{snugshade}}{\end{snugshade}}
\newcommand{\KeywordTok}[1]{\textcolor[rgb]{0.13,0.29,0.53}{\textbf{#1}}}
\newcommand{\DataTypeTok}[1]{\textcolor[rgb]{0.13,0.29,0.53}{#1}}
\newcommand{\DecValTok}[1]{\textcolor[rgb]{0.00,0.00,0.81}{#1}}
\newcommand{\BaseNTok}[1]{\textcolor[rgb]{0.00,0.00,0.81}{#1}}
\newcommand{\FloatTok}[1]{\textcolor[rgb]{0.00,0.00,0.81}{#1}}
\newcommand{\ConstantTok}[1]{\textcolor[rgb]{0.00,0.00,0.00}{#1}}
\newcommand{\CharTok}[1]{\textcolor[rgb]{0.31,0.60,0.02}{#1}}
\newcommand{\SpecialCharTok}[1]{\textcolor[rgb]{0.00,0.00,0.00}{#1}}
\newcommand{\StringTok}[1]{\textcolor[rgb]{0.31,0.60,0.02}{#1}}
\newcommand{\VerbatimStringTok}[1]{\textcolor[rgb]{0.31,0.60,0.02}{#1}}
\newcommand{\SpecialStringTok}[1]{\textcolor[rgb]{0.31,0.60,0.02}{#1}}
\newcommand{\ImportTok}[1]{#1}
\newcommand{\CommentTok}[1]{\textcolor[rgb]{0.56,0.35,0.01}{\textit{#1}}}
\newcommand{\DocumentationTok}[1]{\textcolor[rgb]{0.56,0.35,0.01}{\textbf{\textit{#1}}}}
\newcommand{\AnnotationTok}[1]{\textcolor[rgb]{0.56,0.35,0.01}{\textbf{\textit{#1}}}}
\newcommand{\CommentVarTok}[1]{\textcolor[rgb]{0.56,0.35,0.01}{\textbf{\textit{#1}}}}
\newcommand{\OtherTok}[1]{\textcolor[rgb]{0.56,0.35,0.01}{#1}}
\newcommand{\FunctionTok}[1]{\textcolor[rgb]{0.00,0.00,0.00}{#1}}
\newcommand{\VariableTok}[1]{\textcolor[rgb]{0.00,0.00,0.00}{#1}}
\newcommand{\ControlFlowTok}[1]{\textcolor[rgb]{0.13,0.29,0.53}{\textbf{#1}}}
\newcommand{\OperatorTok}[1]{\textcolor[rgb]{0.81,0.36,0.00}{\textbf{#1}}}
\newcommand{\BuiltInTok}[1]{#1}
\newcommand{\ExtensionTok}[1]{#1}
\newcommand{\PreprocessorTok}[1]{\textcolor[rgb]{0.56,0.35,0.01}{\textit{#1}}}
\newcommand{\AttributeTok}[1]{\textcolor[rgb]{0.77,0.63,0.00}{#1}}
\newcommand{\RegionMarkerTok}[1]{#1}
\newcommand{\InformationTok}[1]{\textcolor[rgb]{0.56,0.35,0.01}{\textbf{\textit{#1}}}}
\newcommand{\WarningTok}[1]{\textcolor[rgb]{0.56,0.35,0.01}{\textbf{\textit{#1}}}}
\newcommand{\AlertTok}[1]{\textcolor[rgb]{0.94,0.16,0.16}{#1}}
\newcommand{\ErrorTok}[1]{\textcolor[rgb]{0.64,0.00,0.00}{\textbf{#1}}}
\newcommand{\NormalTok}[1]{#1}
\usepackage{graphicx,grffile}
\makeatletter
\def\maxwidth{\ifdim\Gin@nat@width>\linewidth\linewidth\else\Gin@nat@width\fi}
\def\maxheight{\ifdim\Gin@nat@height>\textheight\textheight\else\Gin@nat@height\fi}
\makeatother
% Scale images if necessary, so that they will not overflow the page
% margins by default, and it is still possible to overwrite the defaults
% using explicit options in \includegraphics[width, height, ...]{}
\setkeys{Gin}{width=\maxwidth,height=\maxheight,keepaspectratio}
\IfFileExists{parskip.sty}{%
\usepackage{parskip}
}{% else
\setlength{\parindent}{0pt}
\setlength{\parskip}{6pt plus 2pt minus 1pt}
}
\setlength{\emergencystretch}{3em}  % prevent overfull lines
\providecommand{\tightlist}{%
  \setlength{\itemsep}{0pt}\setlength{\parskip}{0pt}}
\setcounter{secnumdepth}{0}
% Redefines (sub)paragraphs to behave more like sections
\ifx\paragraph\undefined\else
\let\oldparagraph\paragraph
\renewcommand{\paragraph}[1]{\oldparagraph{#1}\mbox{}}
\fi
\ifx\subparagraph\undefined\else
\let\oldsubparagraph\subparagraph
\renewcommand{\subparagraph}[1]{\oldsubparagraph{#1}\mbox{}}
\fi

%%% Use protect on footnotes to avoid problems with footnotes in titles
\let\rmarkdownfootnote\footnote%
\def\footnote{\protect\rmarkdownfootnote}

%%% Change title format to be more compact
\usepackage{titling}

% Create subtitle command for use in maketitle
\newcommand{\subtitle}[1]{
  \posttitle{
    \begin{center}\large#1\end{center}
    }
}

\setlength{\droptitle}{-2em}

  \title{R tutorial}
    \pretitle{\vspace{\droptitle}\centering\huge}
  \posttitle{\par}
  \subtitle{Econometrics 322}
  \author{Hang Miao}
    \preauthor{\centering\large\emph}
  \postauthor{\par}
    \date{}
    \predate{}\postdate{}
  

\begin{document}
\maketitle

\section{\texorpdfstring{0. \textbf{Assignment and
Basics}}{0. Assignment and Basics}}\label{assignment-and-basics}

\begin{Shaded}
\begin{Highlighting}[]
\NormalTok{n <-}\StringTok{ }\DecValTok{15}
\NormalTok{n}
\end{Highlighting}
\end{Shaded}

\begin{verbatim}
## [1] 15
\end{verbatim}

\begin{Shaded}
\begin{Highlighting}[]
\NormalTok{a =}\StringTok{ }\DecValTok{12}
\NormalTok{a}
\end{Highlighting}
\end{Shaded}

\begin{verbatim}
## [1] 12
\end{verbatim}

\begin{Shaded}
\begin{Highlighting}[]
\DecValTok{24}\NormalTok{ ->}\StringTok{ }\NormalTok{z}
\NormalTok{z}
\end{Highlighting}
\end{Shaded}

\begin{verbatim}
## [1] 24
\end{verbatim}

\subsubsection{Variables must start with a letter, but may also contain
numbers and periods. R is case
sensitive.}\label{variables-must-start-with-a-letter-but-may-also-contain-numbers-and-periods.-r-is-case-sensitive.}

\begin{Shaded}
\begin{Highlighting}[]
\NormalTok{N <-}\StringTok{ }\FloatTok{26.42}
\NormalTok{N}
\end{Highlighting}
\end{Shaded}

\begin{verbatim}
## [1] 26.42
\end{verbatim}

\begin{Shaded}
\begin{Highlighting}[]
\NormalTok{n}
\end{Highlighting}
\end{Shaded}

\begin{verbatim}
## [1] 15
\end{verbatim}

\subsubsection{To see a list of your objects, use ls( ). The ( ) is
required, even though there are no
arguments.}\label{to-see-a-list-of-your-objects-use-ls-.-the-is-required-even-though-there-are-no-arguments.}

\begin{Shaded}
\begin{Highlighting}[]
\KeywordTok{ls}\NormalTok{()}
\end{Highlighting}
\end{Shaded}

\begin{verbatim}
## [1] "a" "n" "N" "z"
\end{verbatim}

\subsubsection{Use rm to delete objects you no longer
need.}\label{use-rm-to-delete-objects-you-no-longer-need.}

\begin{Shaded}
\begin{Highlighting}[]
\KeywordTok{rm}\NormalTok{(n)}
\end{Highlighting}
\end{Shaded}

\begin{Shaded}
\begin{Highlighting}[]
\KeywordTok{ls}\NormalTok{()}
\end{Highlighting}
\end{Shaded}

\begin{verbatim}
## [1] "a" "N" "z"
\end{verbatim}

\subsubsection{You may see online help about a function using the help
command or a question
mark.}\label{you-may-see-online-help-about-a-function-using-the-help-command-or-a-question-mark.}

\begin{Shaded}
\begin{Highlighting}[]
\NormalTok{?ls}
\end{Highlighting}
\end{Shaded}

\begin{Shaded}
\begin{Highlighting}[]
\KeywordTok{help}\NormalTok{(rm)}
\end{Highlighting}
\end{Shaded}

\subsubsection{Several commands are available to help find a command
whose name you don't know. Note that anything after a pound sign (\#) is
a comment and will not have any effect on
R.}\label{several-commands-are-available-to-help-find-a-command-whose-name-you-dont-know.-note-that-anything-after-a-pound-sign-is-a-comment-and-will-not-have-any-effect-on-r.}

\begin{Shaded}
\begin{Highlighting}[]
\KeywordTok{help.search}\NormalTok{(}\StringTok{"help"}\NormalTok{) }\CommentTok{# "help" in name or summary; note quotes!}
\end{Highlighting}
\end{Shaded}

\begin{Shaded}
\begin{Highlighting}[]
\KeywordTok{help.start}\NormalTok{() }\CommentTok{# also remember the R Commands web page}
\end{Highlighting}
\end{Shaded}

\begin{verbatim}
## starting httpd help server ... done
\end{verbatim}

\begin{verbatim}
## If the browser launched by '/usr/bin/open' is already running, it
##     is *not* restarted, and you must switch to its window.
## Otherwise, be patient ...
\end{verbatim}

\subsubsection{Other data types are available. You do not need to
declare these; they will be assigned
automatically.}\label{other-data-types-are-available.-you-do-not-need-to-declare-these-they-will-be-assigned-automatically.}

name \textless{}- ``Mike'' \# Character data name

\begin{Shaded}
\begin{Highlighting}[]
\NormalTok{q1 <-}\StringTok{ }\OtherTok{TRUE} \CommentTok{# Logical data}
\NormalTok{q1}
\end{Highlighting}
\end{Shaded}

\begin{verbatim}
## [1] TRUE
\end{verbatim}

\begin{Shaded}
\begin{Highlighting}[]
\NormalTok{q2 <-}\StringTok{ }\NormalTok{F}
\NormalTok{q2}
\end{Highlighting}
\end{Shaded}

\begin{verbatim}
## [1] FALSE
\end{verbatim}

\section{\texorpdfstring{1. \textbf{Simple
calculation}}{1. Simple calculation}}\label{simple-calculation}

\subsubsection{R may be used for simple calculation, using the standard
arithmetic symbols +, -, *, /, as well as parentheses and \^{}
(exponentiation).}\label{r-may-be-used-for-simple-calculation-using-the-standard-arithmetic-symbols---as-well-as-parentheses-and-exponentiation.}

\begin{Shaded}
\begin{Highlighting}[]
\NormalTok{a <-}\StringTok{ }\DecValTok{12}\OperatorTok{+}\DecValTok{14}
\NormalTok{a}
\end{Highlighting}
\end{Shaded}

\begin{verbatim}
## [1] 26
\end{verbatim}

\begin{Shaded}
\begin{Highlighting}[]
\DecValTok{3}\OperatorTok{*}\DecValTok{5}
\end{Highlighting}
\end{Shaded}

\begin{verbatim}
## [1] 15
\end{verbatim}

\begin{Shaded}
\begin{Highlighting}[]
\NormalTok{(}\DecValTok{20}\OperatorTok{-}\DecValTok{4}\NormalTok{)}\OperatorTok{/}\DecValTok{2}
\end{Highlighting}
\end{Shaded}

\begin{verbatim}
## [1] 8
\end{verbatim}

\begin{Shaded}
\begin{Highlighting}[]
\DecValTok{7}\OperatorTok{^}\DecValTok{2}
\end{Highlighting}
\end{Shaded}

\begin{verbatim}
## [1] 49
\end{verbatim}

\subsubsection{Standard mathematical functions are
available.}\label{standard-mathematical-functions-are-available.}

\begin{Shaded}
\begin{Highlighting}[]
\KeywordTok{exp}\NormalTok{(}\DecValTok{2}\NormalTok{)}
\end{Highlighting}
\end{Shaded}

\begin{verbatim}
## [1] 7.389056
\end{verbatim}

\begin{Shaded}
\begin{Highlighting}[]
\KeywordTok{log}\NormalTok{(}\DecValTok{10}\NormalTok{) }\CommentTok{# Natural log}
\end{Highlighting}
\end{Shaded}

\begin{verbatim}
## [1] 2.302585
\end{verbatim}

\begin{Shaded}
\begin{Highlighting}[]
\KeywordTok{log10}\NormalTok{(}\DecValTok{10}\NormalTok{) }\CommentTok{# Base 10}
\end{Highlighting}
\end{Shaded}

\begin{verbatim}
## [1] 1
\end{verbatim}

\begin{Shaded}
\begin{Highlighting}[]
\KeywordTok{log2}\NormalTok{(}\DecValTok{64}\NormalTok{) }\CommentTok{# Base 2}
\end{Highlighting}
\end{Shaded}

\begin{verbatim}
## [1] 6
\end{verbatim}

\begin{Shaded}
\begin{Highlighting}[]
\NormalTok{pi}
\end{Highlighting}
\end{Shaded}

\begin{verbatim}
## [1] 3.141593
\end{verbatim}

\begin{Shaded}
\begin{Highlighting}[]
\KeywordTok{cos}\NormalTok{(pi)}
\end{Highlighting}
\end{Shaded}

\begin{verbatim}
## [1] -1
\end{verbatim}

\begin{Shaded}
\begin{Highlighting}[]
\KeywordTok{sqrt}\NormalTok{(}\DecValTok{100}\NormalTok{) }\CommentTok{# square root}
\end{Highlighting}
\end{Shaded}

\begin{verbatim}
## [1] 10
\end{verbatim}

\section{\texorpdfstring{\textbf{2.
Vectors}}{2. Vectors}}\label{vectors}

\subsubsection{Vectors may be created using the c command, separating
your elements with
commas.}\label{vectors-may-be-created-using-the-c-command-separating-your-elements-with-commas.}

\begin{Shaded}
\begin{Highlighting}[]
\NormalTok{a <-}\StringTok{ }\KeywordTok{c}\NormalTok{(}\DecValTok{1}\NormalTok{, }\DecValTok{7}\NormalTok{, }\DecValTok{32}\NormalTok{, }\DecValTok{16}\NormalTok{)}
\NormalTok{a}
\end{Highlighting}
\end{Shaded}

\begin{verbatim}
## [1]  1  7 32 16
\end{verbatim}

\subsubsection{Sequences of integers may be created using a colon
(:).}\label{sequences-of-integers-may-be-created-using-a-colon-.}

\begin{Shaded}
\begin{Highlighting}[]
\NormalTok{b <-}\StringTok{ }\DecValTok{1}\OperatorTok{:}\DecValTok{10}
\NormalTok{b}
\end{Highlighting}
\end{Shaded}

\begin{verbatim}
##  [1]  1  2  3  4  5  6  7  8  9 10
\end{verbatim}

\begin{Shaded}
\begin{Highlighting}[]
\NormalTok{c <-}\StringTok{ }\DecValTok{20}\OperatorTok{:}\DecValTok{15}
\NormalTok{c}
\end{Highlighting}
\end{Shaded}

\begin{verbatim}
## [1] 20 19 18 17 16 15
\end{verbatim}

\subsubsection{Other regular vectors may be created using the seq
(sequence) and rep (repeat)
commands.}\label{other-regular-vectors-may-be-created-using-the-seq-sequence-and-rep-repeat-commands.}

\begin{Shaded}
\begin{Highlighting}[]
\NormalTok{d <-}\StringTok{ }\KeywordTok{seq}\NormalTok{(}\DecValTok{1}\NormalTok{, }\DecValTok{5}\NormalTok{, }\DataTypeTok{by=}\FloatTok{0.5}\NormalTok{)}
\NormalTok{d}
\end{Highlighting}
\end{Shaded}

\begin{verbatim}
## [1] 1.0 1.5 2.0 2.5 3.0 3.5 4.0 4.5 5.0
\end{verbatim}

\begin{Shaded}
\begin{Highlighting}[]
\NormalTok{e <-}\StringTok{ }\KeywordTok{seq}\NormalTok{(}\DecValTok{0}\NormalTok{, }\DecValTok{10}\NormalTok{, }\DataTypeTok{length=}\DecValTok{5}\NormalTok{)}
\NormalTok{e}
\end{Highlighting}
\end{Shaded}

\begin{verbatim}
## [1]  0.0  2.5  5.0  7.5 10.0
\end{verbatim}

\begin{Shaded}
\begin{Highlighting}[]
\NormalTok{f <-}\StringTok{ }\KeywordTok{rep}\NormalTok{(}\DecValTok{0}\NormalTok{, }\DecValTok{5}\NormalTok{)}
\NormalTok{f}
\end{Highlighting}
\end{Shaded}

\begin{verbatim}
## [1] 0 0 0 0 0
\end{verbatim}

\begin{Shaded}
\begin{Highlighting}[]
\NormalTok{g <-}\StringTok{ }\KeywordTok{rep}\NormalTok{(}\DecValTok{1}\OperatorTok{:}\DecValTok{3}\NormalTok{, }\DecValTok{4}\NormalTok{)}
\NormalTok{g}
\end{Highlighting}
\end{Shaded}

\begin{verbatim}
##  [1] 1 2 3 1 2 3 1 2 3 1 2 3
\end{verbatim}

\begin{Shaded}
\begin{Highlighting}[]
\NormalTok{h <-}\StringTok{ }\KeywordTok{rep}\NormalTok{(}\DecValTok{4}\OperatorTok{:}\DecValTok{6}\NormalTok{, }\DecValTok{1}\OperatorTok{:}\DecValTok{3}\NormalTok{)}
\NormalTok{h}
\end{Highlighting}
\end{Shaded}

\begin{verbatim}
## [1] 4 5 5 6 6 6
\end{verbatim}

\subsubsection{Random vectors can be created with a set of functions
that start with r, such as rnorm (normal) or runif
(uniform).}\label{random-vectors-can-be-created-with-a-set-of-functions-that-start-with-r-such-as-rnorm-normal-or-runif-uniform.}

\begin{Shaded}
\begin{Highlighting}[]
\NormalTok{x <-}\StringTok{ }\KeywordTok{rnorm}\NormalTok{(}\DecValTok{5}\NormalTok{) }\CommentTok{# Standard normal random variables}
\NormalTok{x}
\end{Highlighting}
\end{Shaded}

\begin{verbatim}
## [1] -0.3285511  0.5440755  1.1188503  1.2524622  0.5721741
\end{verbatim}

\begin{Shaded}
\begin{Highlighting}[]
\NormalTok{y <-}\StringTok{ }\KeywordTok{rnorm}\NormalTok{(}\DecValTok{7}\NormalTok{, }\DecValTok{10}\NormalTok{, }\DecValTok{3}\NormalTok{) }\CommentTok{# Normal r.v.s with mu = 10, sigma = 3}
\NormalTok{y}
\end{Highlighting}
\end{Shaded}

\begin{verbatim}
## [1]  5.429975 10.097136 14.642965  9.096982 11.271690 11.585727 12.916732
\end{verbatim}

\begin{Shaded}
\begin{Highlighting}[]
\NormalTok{z <-}\StringTok{ }\KeywordTok{runif}\NormalTok{(}\DecValTok{10}\NormalTok{) }\CommentTok{# Uniform(0, 1) random variables}
\NormalTok{z}
\end{Highlighting}
\end{Shaded}

\begin{verbatim}
##  [1] 0.9056515 0.7958397 0.7443220 0.7062697 0.7093662 0.7533793 0.3966831
##  [8] 0.1803170 0.1504333 0.4823195
\end{verbatim}

\subsubsection{If a vector is passed to an arithmetic calculation, it
will be computed
element-by-element.}\label{if-a-vector-is-passed-to-an-arithmetic-calculation-it-will-be-computed-element-by-element.}

\begin{Shaded}
\begin{Highlighting}[]
\KeywordTok{c}\NormalTok{(}\DecValTok{1}\NormalTok{, }\DecValTok{2}\NormalTok{, }\DecValTok{3}\NormalTok{) }\OperatorTok{+}\StringTok{ }\KeywordTok{c}\NormalTok{(}\DecValTok{4}\NormalTok{, }\DecValTok{5}\NormalTok{, }\DecValTok{6}\NormalTok{)}
\end{Highlighting}
\end{Shaded}

\begin{verbatim}
## [1] 5 7 9
\end{verbatim}

\begin{Shaded}
\begin{Highlighting}[]
\KeywordTok{sqrt}\NormalTok{(}\KeywordTok{c}\NormalTok{(}\DecValTok{100}\NormalTok{, }\DecValTok{225}\NormalTok{, }\DecValTok{400}\NormalTok{))}
\end{Highlighting}
\end{Shaded}

\begin{verbatim}
## [1] 10 15 20
\end{verbatim}

\subsubsection{If the vectors involved are of different lengths, the
shorter one will be repeated until it is the same length as the
longer.}\label{if-the-vectors-involved-are-of-different-lengths-the-shorter-one-will-be-repeated-until-it-is-the-same-length-as-the-longer.}

\begin{Shaded}
\begin{Highlighting}[]
\KeywordTok{c}\NormalTok{(}\DecValTok{1}\NormalTok{, }\DecValTok{2}\NormalTok{, }\DecValTok{3}\NormalTok{, }\DecValTok{4}\NormalTok{) }\OperatorTok{+}\StringTok{ }\KeywordTok{c}\NormalTok{(}\DecValTok{10}\NormalTok{, }\DecValTok{20}\NormalTok{)}
\end{Highlighting}
\end{Shaded}

\begin{verbatim}
## [1] 11 22 13 24
\end{verbatim}

\begin{Shaded}
\begin{Highlighting}[]
\KeywordTok{c}\NormalTok{(}\DecValTok{1}\NormalTok{, }\DecValTok{2}\NormalTok{, }\DecValTok{3}\NormalTok{) }\OperatorTok{+}\StringTok{ }\KeywordTok{c}\NormalTok{(}\DecValTok{10}\NormalTok{, }\DecValTok{20}\NormalTok{)}
\end{Highlighting}
\end{Shaded}

\begin{verbatim}
## Warning in c(1, 2, 3) + c(10, 20): longer object length is not a multiple
## of shorter object length
\end{verbatim}

\begin{verbatim}
## [1] 11 22 13
\end{verbatim}

\subsubsection{To select subsets of a vector, use square brackets ({[}
{]}).}\label{to-select-subsets-of-a-vector-use-square-brackets-.}

\begin{Shaded}
\begin{Highlighting}[]
\NormalTok{d}
\end{Highlighting}
\end{Shaded}

\begin{verbatim}
## [1] 1.0 1.5 2.0 2.5 3.0 3.5 4.0 4.5 5.0
\end{verbatim}

\begin{Shaded}
\begin{Highlighting}[]
\NormalTok{d[}\DecValTok{3}\NormalTok{]}
\end{Highlighting}
\end{Shaded}

\begin{verbatim}
## [1] 2
\end{verbatim}

\begin{Shaded}
\begin{Highlighting}[]
\NormalTok{d[}\DecValTok{5}\OperatorTok{:}\DecValTok{7}\NormalTok{]}
\end{Highlighting}
\end{Shaded}

\begin{verbatim}
## [1] 3.0 3.5 4.0
\end{verbatim}

\subsubsection{A logical vector in the brackets will return the TRUE
elements.}\label{a-logical-vector-in-the-brackets-will-return-the-true-elements.}

\begin{Shaded}
\begin{Highlighting}[]
\NormalTok{d }\OperatorTok{>}\StringTok{ }\FloatTok{2.8}
\end{Highlighting}
\end{Shaded}

\begin{verbatim}
## [1] FALSE FALSE FALSE FALSE  TRUE  TRUE  TRUE  TRUE  TRUE
\end{verbatim}

\begin{Shaded}
\begin{Highlighting}[]
\NormalTok{d[d }\OperatorTok{>}\StringTok{ }\FloatTok{2.8}\NormalTok{]}
\end{Highlighting}
\end{Shaded}

\begin{verbatim}
## [1] 3.0 3.5 4.0 4.5 5.0
\end{verbatim}

\subsubsection{The number of elements in a vector can be found with the
length
command.}\label{the-number-of-elements-in-a-vector-can-be-found-with-the-length-command.}

\begin{Shaded}
\begin{Highlighting}[]
\KeywordTok{length}\NormalTok{(d)}
\end{Highlighting}
\end{Shaded}

\begin{verbatim}
## [1] 9
\end{verbatim}

\begin{Shaded}
\begin{Highlighting}[]
\KeywordTok{length}\NormalTok{(d[d }\OperatorTok{>}\StringTok{ }\FloatTok{2.8}\NormalTok{])}
\end{Highlighting}
\end{Shaded}

\begin{verbatim}
## [1] 5
\end{verbatim}

\section{\texorpdfstring{\textbf{3. Simple
statistics}}{3. Simple statistics}}\label{simple-statistics}

\subsubsection{There are a variety of mathematical and statistical
summaries which can be computed from a
vector.}\label{there-are-a-variety-of-mathematical-and-statistical-summaries-which-can-be-computed-from-a-vector.}

\begin{Shaded}
\begin{Highlighting}[]
\DecValTok{1}\OperatorTok{:}\DecValTok{4}
\end{Highlighting}
\end{Shaded}

\begin{verbatim}
## [1] 1 2 3 4
\end{verbatim}

\begin{Shaded}
\begin{Highlighting}[]
\KeywordTok{sum}\NormalTok{(}\DecValTok{1}\OperatorTok{:}\DecValTok{4}\NormalTok{)}
\end{Highlighting}
\end{Shaded}

\begin{verbatim}
## [1] 10
\end{verbatim}

\begin{Shaded}
\begin{Highlighting}[]
\KeywordTok{prod}\NormalTok{(}\DecValTok{1}\OperatorTok{:}\DecValTok{4}\NormalTok{) }\CommentTok{# product}
\end{Highlighting}
\end{Shaded}

\begin{verbatim}
## [1] 24
\end{verbatim}

\begin{Shaded}
\begin{Highlighting}[]
\DecValTok{24}
\end{Highlighting}
\end{Shaded}

\begin{verbatim}
## [1] 24
\end{verbatim}

\begin{Shaded}
\begin{Highlighting}[]
\KeywordTok{max}\NormalTok{(}\DecValTok{1}\OperatorTok{:}\DecValTok{10}\NormalTok{)}
\end{Highlighting}
\end{Shaded}

\begin{verbatim}
## [1] 10
\end{verbatim}

\begin{Shaded}
\begin{Highlighting}[]
\KeywordTok{min}\NormalTok{(}\DecValTok{1}\OperatorTok{:}\DecValTok{10}\NormalTok{)}
\end{Highlighting}
\end{Shaded}

\begin{verbatim}
## [1] 1
\end{verbatim}

\begin{Shaded}
\begin{Highlighting}[]
\KeywordTok{range}\NormalTok{(}\DecValTok{1}\OperatorTok{:}\DecValTok{10}\NormalTok{)}
\end{Highlighting}
\end{Shaded}

\begin{verbatim}
## [1]  1 10
\end{verbatim}

\begin{Shaded}
\begin{Highlighting}[]
\NormalTok{X <-}\StringTok{ }\KeywordTok{rnorm}\NormalTok{(}\DecValTok{10}\NormalTok{)}
\NormalTok{X}
\end{Highlighting}
\end{Shaded}

\begin{verbatim}
##  [1]  1.1309652 -0.5398528 -0.8574352  1.9165319  0.8138337 -1.2700377
##  [7]  0.5860495  0.3706055  0.4214494 -1.5509829
\end{verbatim}

\begin{Shaded}
\begin{Highlighting}[]
\KeywordTok{mean}\NormalTok{(X)}
\end{Highlighting}
\end{Shaded}

\begin{verbatim}
## [1] 0.1021127
\end{verbatim}

\begin{Shaded}
\begin{Highlighting}[]
\KeywordTok{sort}\NormalTok{(X)}
\end{Highlighting}
\end{Shaded}

\begin{verbatim}
##  [1] -1.5509829 -1.2700377 -0.8574352 -0.5398528  0.3706055  0.4214494
##  [7]  0.5860495  0.8138337  1.1309652  1.9165319
\end{verbatim}

\begin{Shaded}
\begin{Highlighting}[]
\KeywordTok{median}\NormalTok{(X)}
\end{Highlighting}
\end{Shaded}

\begin{verbatim}
## [1] 0.3960275
\end{verbatim}

\begin{Shaded}
\begin{Highlighting}[]
\KeywordTok{var}\NormalTok{(X)}
\end{Highlighting}
\end{Shaded}

\begin{verbatim}
## [1] 1.245982
\end{verbatim}

\begin{Shaded}
\begin{Highlighting}[]
\KeywordTok{sd}\NormalTok{(X)}
\end{Highlighting}
\end{Shaded}

\begin{verbatim}
## [1] 1.116235
\end{verbatim}

\section{** 4. Matrices **}\label{matrices}

\subsubsection{Matrices can be created with the matrix command,
specifying all elements (column-by-column) as well as the number of rows
and number of
columns.}\label{matrices-can-be-created-with-the-matrix-command-specifying-all-elements-column-by-column-as-well-as-the-number-of-rows-and-number-of-columns.}

\begin{Shaded}
\begin{Highlighting}[]
\NormalTok{A <-}\StringTok{ }\KeywordTok{matrix}\NormalTok{(}\DecValTok{1}\OperatorTok{:}\DecValTok{12}\NormalTok{, }\DataTypeTok{nr=}\DecValTok{3}\NormalTok{, }\DataTypeTok{nc=}\DecValTok{4}\NormalTok{)}
\NormalTok{A}
\end{Highlighting}
\end{Shaded}

\begin{verbatim}
##      [,1] [,2] [,3] [,4]
## [1,]    1    4    7   10
## [2,]    2    5    8   11
## [3,]    3    6    9   12
\end{verbatim}

\subsubsection{You may also specify the rows (or columns) as vectors,
and then combine them into a matrix using the rbind (cbind)
command.}\label{you-may-also-specify-the-rows-or-columns-as-vectors-and-then-combine-them-into-a-matrix-using-the-rbind-cbind-command.}

\begin{Shaded}
\begin{Highlighting}[]
\NormalTok{a <-}\StringTok{ }\KeywordTok{c}\NormalTok{(}\DecValTok{1}\NormalTok{,}\DecValTok{2}\NormalTok{,}\DecValTok{3}\NormalTok{)}
\NormalTok{a}
\end{Highlighting}
\end{Shaded}

\begin{verbatim}
## [1] 1 2 3
\end{verbatim}

\begin{Shaded}
\begin{Highlighting}[]
\NormalTok{b <-}\StringTok{ }\KeywordTok{c}\NormalTok{(}\DecValTok{10}\NormalTok{, }\DecValTok{20}\NormalTok{, }\DecValTok{30}\NormalTok{)}
\NormalTok{b}
\end{Highlighting}
\end{Shaded}

\begin{verbatim}
## [1] 10 20 30
\end{verbatim}

\begin{Shaded}
\begin{Highlighting}[]
\NormalTok{c <-}\StringTok{ }\KeywordTok{c}\NormalTok{(}\DecValTok{100}\NormalTok{, }\DecValTok{200}\NormalTok{, }\DecValTok{300}\NormalTok{)}
\NormalTok{c}
\end{Highlighting}
\end{Shaded}

\begin{verbatim}
## [1] 100 200 300
\end{verbatim}

\begin{Shaded}
\begin{Highlighting}[]
\NormalTok{d <-}\StringTok{ }\KeywordTok{c}\NormalTok{(}\DecValTok{1000}\NormalTok{, }\DecValTok{2000}\NormalTok{, }\DecValTok{3000}\NormalTok{)}
\NormalTok{d}
\end{Highlighting}
\end{Shaded}

\begin{verbatim}
## [1] 1000 2000 3000
\end{verbatim}

\begin{Shaded}
\begin{Highlighting}[]
\NormalTok{B <-}\StringTok{ }\KeywordTok{rbind}\NormalTok{(a, b, c, d)}
\NormalTok{B}
\end{Highlighting}
\end{Shaded}

\begin{verbatim}
##   [,1] [,2] [,3]
## a    1    2    3
## b   10   20   30
## c  100  200  300
## d 1000 2000 3000
\end{verbatim}

\begin{Shaded}
\begin{Highlighting}[]
\NormalTok{C <-}\StringTok{ }\KeywordTok{cbind}\NormalTok{(a, b, c, d)}
\NormalTok{C}
\end{Highlighting}
\end{Shaded}

\begin{verbatim}
##      a  b   c    d
## [1,] 1 10 100 1000
## [2,] 2 20 200 2000
## [3,] 3 30 300 3000
\end{verbatim}

\subsubsection{To select a subset of a matrix, use the square brackets
and specify rows before the comma, and columns
after.}\label{to-select-a-subset-of-a-matrix-use-the-square-brackets-and-specify-rows-before-the-comma-and-columns-after.}

\begin{Shaded}
\begin{Highlighting}[]
\NormalTok{C[}\DecValTok{1}\OperatorTok{:}\DecValTok{2}\NormalTok{,]}
\end{Highlighting}
\end{Shaded}

\begin{verbatim}
##      a  b   c    d
## [1,] 1 10 100 1000
## [2,] 2 20 200 2000
\end{verbatim}

\begin{Shaded}
\begin{Highlighting}[]
\NormalTok{C[,}\KeywordTok{c}\NormalTok{(}\DecValTok{1}\NormalTok{,}\DecValTok{3}\NormalTok{)]}
\end{Highlighting}
\end{Shaded}

\begin{verbatim}
##      a   c
## [1,] 1 100
## [2,] 2 200
## [3,] 3 300
\end{verbatim}

\begin{Shaded}
\begin{Highlighting}[]
\NormalTok{C[}\DecValTok{1}\OperatorTok{:}\DecValTok{2}\NormalTok{,}\KeywordTok{c}\NormalTok{(}\DecValTok{1}\NormalTok{,}\DecValTok{3}\NormalTok{)]}
\end{Highlighting}
\end{Shaded}

\begin{verbatim}
##      a   c
## [1,] 1 100
## [2,] 2 200
\end{verbatim}

\subsubsection{Matrix multiplication is performed with the operator
\%*\%. Remember that order
matters!}\label{matrix-multiplication-is-performed-with-the-operator-.-remember-that-order-matters}

\begin{Shaded}
\begin{Highlighting}[]
\NormalTok{B}\OperatorTok\NormalTok{C}
\end{Highlighting}
\end{Shaded}

\begin{verbatim}
##       a      b       c       d
## a    14    140    1400 1.4e+04
## b   140   1400   14000 1.4e+05
## c  1400  14000  140000 1.4e+06
## d 14000 140000 1400000 1.4e+07
\end{verbatim}

\begin{Shaded}
\begin{Highlighting}[]
\NormalTok{C}\OperatorTok\NormalTok{B}
\end{Highlighting}
\end{Shaded}

\begin{verbatim}
##         [,1]    [,2]    [,3]
## [1,] 1010101 2020202 3030303
## [2,] 2020202 4040404 6060606
## [3,] 3030303 6060606 9090909
\end{verbatim}

\subsubsection{You may apply a summary function to the rows or columns
of a matrix using the apply
function.}\label{you-may-apply-a-summary-function-to-the-rows-or-columns-of-a-matrix-using-the-apply-function.}

\begin{Shaded}
\begin{Highlighting}[]
\NormalTok{C}
\end{Highlighting}
\end{Shaded}

\begin{verbatim}
##      a  b   c    d
## [1,] 1 10 100 1000
## [2,] 2 20 200 2000
## [3,] 3 30 300 3000
\end{verbatim}

\begin{Shaded}
\begin{Highlighting}[]
\KeywordTok{sum}\NormalTok{(C)}
\end{Highlighting}
\end{Shaded}

\begin{verbatim}
## [1] 6666
\end{verbatim}

\begin{Shaded}
\begin{Highlighting}[]
\KeywordTok{apply}\NormalTok{(C, }\DecValTok{1}\NormalTok{, sum) }\CommentTok{# apply sum function on each row}
\end{Highlighting}
\end{Shaded}

\begin{verbatim}
## [1] 1111 2222 3333
\end{verbatim}

\begin{Shaded}
\begin{Highlighting}[]
\KeywordTok{apply}\NormalTok{(C, }\DecValTok{2}\NormalTok{, sum) }\CommentTok{# apply sum function on each column}
\end{Highlighting}
\end{Shaded}

\begin{verbatim}
##    a    b    c    d 
##    6   60  600 6000
\end{verbatim}

\begin{Shaded}
\begin{Highlighting}[]
\KeywordTok{rowSums}\NormalTok{(C)}
\end{Highlighting}
\end{Shaded}

\begin{verbatim}
## [1] 1111 2222 3333
\end{verbatim}

\begin{Shaded}
\begin{Highlighting}[]
\KeywordTok{colSums}\NormalTok{(C)}
\end{Highlighting}
\end{Shaded}

\begin{verbatim}
##    a    b    c    d 
##    6   60  600 6000
\end{verbatim}


\end{document}
